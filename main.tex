\documentclass[a4paper,12pt]{article}
\usepackage{mypreamble}
\usepackage{import}
%%%%%% Nomainīt datus %%%%%%%%%%
\author{Vārds Uzvārds}
\title{Nosaukums}
\date{\today}
%%%%%%%%%%%%%%%%%%%%%%%%%%%%%%%%%
\begin{document}
\import{./}{title.tex}                  % Vāks un Titullapa
%\clearpage
\import{./}{headersandfooters.tex}      %   galvene un kājene pārējām lapām
\import{./}{novertejuma_lapa.tex}
\setcounter{page}{2}
\pagebreak
\printglossaries	                    %   saīsinājumi
%\AddToHook{cmd/section/before}{\clearpage} %katra nodaļa sākas jaunā lapas pusē (Ne visos LaTex kompilatoros darbojas- darbojas Overleaf, bet uz privātā datora nedarbojās)
%\setcounter{page}{2} 
\pagebreak
\tableofcontents
\thispagestyle{fancy}                   %  kopējā stila piešķiršana satura lapai
%\setcounter{page}{2}                    %  satura lapas numurs tiek iestatīts 
\pagebreak
\section*{Ievads}
\import{sections/}{ievads.tex}
\pagebreak
\section{Pirmā nodaļa}
\import{sections/}{pirma.tex}
\pagebreak
\section{Otrā nodaļa - attēli}
\import{sections/}{atteli.tex}
\pagebreak
\section{Trešā nodaļa - saīsinājumi}
\import{sections/}{saisinajumi.tex}
\pagebreak
\section{Ceturtā nodaļa - tabulas un formulas}
\import{sections/}{tabulas.tex}
\pagebreak
\section{Piektā nodaļa - literatūra}
\import{sections/}{literatura.tex}
\pagebreak
\section{Sestā nodaļa - pielikumi}
\import{sections/}{pielikumi.tex}
\pagebreak
\import{./}{sections/bibliography.tex}
\pagebreak
% pielikumi
\import{./}{appendices.tex}
\end{document}
