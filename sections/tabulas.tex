\subsection{Tabulas}
\hspace*{5mm} Darbā ievietotās tabulas tiek numurētas, un katrai no tām ir jābūt savam
nosaukumam. Tabulas nosaukumu raksta simetriski virs tabulas ar mazajiem burtiem
un lielo sākuma burtu, bez punkta nosaukuma beigās. Tabulas numurē nodaļas
ietvaros ar arābu cipariem. Tabulas numuru raksta labajā pusē virs tabulas
nosaukuma. Darba ceturtajā nodaļā ievietotas pirmās pēc kārtas tabulas apraksta
piemērs:\\
%
	\begin{table}[!th]
	\begin{center}
		 \begin{tabularx}{\textwidth}{|m{1.5cm}|m{4.8cm}|X|X|}
		  \hline 
		  \textbf{N.p.k.} & \textbf{Saīsinājums} &  \multicolumn{2}{c|}{\textbf{Atšifrējums}} \\ 
		  \hline 
		  \hline 
		 1 & lpp. &  lappuse  & * \\ 
		  \hline 
		  2   & n.p.k. &  numurs pēc kārtas  & * \\ 
		  \hline 
		  3 & sk. vai skat. &  skatīt  & * \\ 
		  \hline 
		  4 & š.g. &  šā gada; šī gada  & * \\ 
		  \hline 
		  5 & t.i. &  tas ir & *  \\ 
		  \hline 
		  6 & u.c. &  un citi & *  \\ 
		  \hline 
		  \end{tabularx}  
		\caption{Man patīk izmantot tabulatix}\label{tab:piemers1}
	\end{center}
	\end{table}
	%
%
	\begin{table}[!th]
	\begin{center}
		 \begin{tabularx}{\textwidth}{|r|c|X|}
		  \hline 
		  \textbf{N.p.k.} & \textbf{Saīsinājums} & 	  \textbf{Atšifrējums} \\ 
		  \hline 
		  \hline 
		 1 & u.tml. &  un tamlīdzīgi   \\ 
		  \hline 
		  2   & utt. &  un tā tālāk  \\ 
		  \hline 
		  \end{tabularx}  
		\caption{Man patīk izmantot tabulatix}\label{tab:piemers2}
	\end{center}
	\end{table}
	%

\subsection{Formulas}
Piemēri:\\
%
\begin{equation}\label{eq:taktis-bita}
	N=\frac{F_{takts}}{R_T}=\frac{100\cdot 10^{6}}{115200}\approx 868,\ [taktis/bits]
\end{equation}
%
\begin{align*} 
	\text{kur }	& N -  \text{takšu skaits cik nepieciešams gaidīt, pirms sākt pārraidīt nākošo bitu, } \\ 
				& F_{takts} -  \text{takts frekvence, } Hz  \\
				& R_T - \text{informācijas pārraides ražīgums, } Bodi
\end{align*}
%
Nākošais piemērs:\\
%
\begin{equation}\label{eq:meroga-matrica}
	T = \begin{bmatrix}
       								0.5 & 0 & 0 \\
      								 0 & 0.5  & 0 \\
      								 0  & 0 & 1
                                        \end{bmatrix} 
\end{equation}
%
\begin{align*} 
	\text{kur }	& T -  \text{transformācijas matrica, kas samazina attēlu uz pusi pa } x \text{ un } y \text{ assi} 
\end{align*}
%